\section{Analysis of the ESBL E. coli patient isolates}

\subsection{SNPs}

We identified several SNPs 	altering genes and promoters for four out of 5 analyzed patients. By studying the literature we try to make connections to the altered genes and other already studied resistance mechanisms. 

\textbf{orT} coding for orphan toxin is an inner-membrane protein which is activated under conditions inducing stringent response \cite{islam_orphan_2015}. By damaging the cell membrane and reducing the intracellular ATP level cell growth is reduced. OrtT can increase persistence, a state in which cells are tolerant to antibiotics \cite{islam_orphan_2015}. Islam et al. showed that expression of ortT increases up to 80 fold when the cells are put under antibiotic pressure. None of the testes antibiotics were \textbeta-lactams. 

\textbf{scrY}


Knowing mutations which play a key factor in gaining resistance would allow to predict resistance of ESBL \textit{E. coli} based on genomic information. 